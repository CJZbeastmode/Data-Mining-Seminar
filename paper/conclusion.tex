
\section{Conclusion}
In a nutshell, SVM is an algorithm that is capable of solving classification
problems with high accuracy. The main idea of SVM is to find the optimal
hyperplane that separates the data into two classes. In cases which the data
points are not linearly separable, SVM is able to transform 
the dataset in a way that they are linearly separable using kernel functions.
It is a highly automated algorithm that does not require users to deal with
the data distributions properties before applying it. However, its flexibility
also allows users to customize their own kernel functions to transform the data
in a way that it suits the real life situation, so that an even better result
could be delivered using this fine-tuning mechanics.

With the properties discussed above, SVM does not really require the users to have deep background
knowledge in the field, since it is highly automated.
The algorithm often delivers high accuracy in binary classification. But it can be modified to deal 
with more complicated problems, such as multi-class classification and
pattern recognition. Its flexibility in usages makes it a popular choice among
researches in many fields. 
Popular applications of SVM includes image preprocessing, authentication, natural language
processing, and bioinformatics.
The drawback of SVM is that the interpretation of the
result might be difficult. Nevertheless, we can use data visualization techniques
to help us understand the relationship between the input and the output.
