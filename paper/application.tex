\section{Application}
With the mathematical definition and deduction being discussed, we can now
look at the application of SVM in real life. Being a classification method,
SVM is widely used in many fields in order to distinguish data classes from
another. In this section, we will discuss five fields that SVM is dominantly
utilized in: Face recognition, text classification, image classification,
bioinformatics and medical diagnosis, as well as handwriting recognition.

\subsection*{Face Recognition}
Face recognition is a biometric method that is used to identify a specific
person. It captures facial features such as eyes, chins, nose, et cetera, and
uses the properties, for example, the distance or the angel between two parts, 
to distinguish one person from another. The difficulty of this method is that
in reality, there are people who look similar to each other. The usage of SVM, 
in this case, strengthens the accuracy by its pattern recognition ability.

Since the basic theory of SVM is used to classify data into two classes, we
need to modify the learning algorithm in order to apply it to the facial
recognition use case. The key here is to combine multiple SVM models in order
to construct a bottom-up binary tree. We compare each pair of nodes in the
tree and the winner of the comparison will be promoted to the upper level.
In the end, the unique class, which is the winner will appear on the top of 
the tree. Given $C$ classes, this algorithm learns $C(C-1)/2$ SVM models at
the training phase and performs $c-1$ comparisons at the testing phase.

This algorithm is used on the Cambridge ORL face database for testing purposes, 
which contains 40 people with 10 face images per person.
The result shows that SVM performs better than other algorithms including
pseudo two-dimensional HMMs (Hidden Markov Model) with a 5\% error rate 
and CNN (Convolutional Neural Network) with a 3.83\% error rate. SVM has
only classified 3.0\% of the images incorrectly. In another experiment,
this algorithm is used to compete against the eigenface method
with nearest center classification (NCC), which has been long applied
in the face recognition field. Nevertheless, SVM still outperformed NCC,
having scored a 8.79\% minimum error rate in comparison to NCC's 15.14\%.
\cite{face-recognition}

\subsection*{Text Classification}
Dealing with text data has long been a challenge in the machine learning
field and this task is no exemption for SVM. In order to suit the algorithm
for text classification, we need to implement pool-based active learning
to the SVM model. The idea is that the learner has access to a pool of
unlabeled data and is allowed to request the label of some data points
in the pool.

Given a set of data points in a multidimensional space, we call the set
of hyperplanes to separate the data points the version space. The goal of
the newly implemented algorithm is to reduce the version space as fast as
possible. Here, we introduce the idea of active learning, which consists of
a boolean classifier, a query strategy and a set of unlabeled data points.
In this case, we can choose the next query in a greedy way such that 
the version space is reduced as fast as possible. There are three ways
to instantiate this idea: 

\begin{itemize}
    \item \textbf{Simple Margin}: The algorithm picks the unlabeled instances
    in the pool whose hyperplane comes closest to the SVM unit vector, which is
    the center of the largest hypersphere that can fit inside the version space.
    \item \textbf{MinMax Margin}: For each unlabeled instance, compute the margins
    of the SVMs obtained when we label the instance as positive and negative
    respectively. Then, we choose the instance with the smallest margin.
    \item \textbf{Ratio Margin}: The algorithm works like MinMax Margin, but uses
    the normalized margin instead of the absolute margin.
\end{itemize}

For inductive learning cases such as email filtering, a classifier is trained 
based on a SVM with a set of labeled data points. 
For transductive learning cases such as relevance feedback, the classifier
learns a SVM with both labeled and unlabeled data points. In reality, it turns
out that combining MinMax margin and ratio margin methods delivers the best
result. \cite{text-classification}

\subsection*{Image Classification}
The problem of object detection in images is a challenging task in the
data mining field. Some of the challenges include: Variations in shape
due to distortions, variations in color due to illumination conditions,
and variations in size due to perspective. 

For this problem, we observe a specific use case: The detection of
license plates in images. The research group in Korea Advanced Institute of 
Science and Technology has proposed an algorithm based on SVM to analyze the 
color textural properties of the license plates. The algorithm goes as
follows:

\begin{itemize}
    \item Set up initial locations and sizes of search windows in images.
    \item Generate the license plate probability distribution using SVM.
    \item Estimate parameters of the license plate using mathematical equations.
    Parameters include: location, size, orientation, and so on.
    \item Modify the search windows based on the estimated parameters.
    \item Reestimate the parameters using the new search windows.
    \item Output bounding windows as the result.
\end{itemize}

This algorithm is called continuously adaptive meanshift algorithm (CAMShift).
The main usage of SVM here is to predict the probability of a pixel belonging
to a license plate, which enables to essential parameters to be accurately estimated.
It produces solid result within seconds, with an average miss rate of around
8.78\% and an average false detection rate of around 14.88\%.
Nevertheless, this algorithm is still not perfectly responsive to bad illumination
and excessive burring.
\cite{image-classification}
