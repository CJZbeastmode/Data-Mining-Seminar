\section{Relative Merits}
Having discussed the application of SVM in real life, we can now conclude
the advantages and disadvantages of SVM as well as the point of using it 
as a classification method. 

The advantages can be described as follows:
\begin{itemize}
    \item \textbf{Suitable for All Data}: The incorporation of kernel functions
    allows SVM to be used on data with any forms of distribution. Kernel functions
    are able to transform the data into a way that it is linearly separable. Therefore,
    one shall not worry about the property of the data while applying SVM.
    \item \textbf{Human Knowledge Not Required}: Since the implementation of kernel functions
    means that SVM is able to deal with any form of data, the user does not need to 
    have prior knowledge about the background of the data. For instance, SVM is easily
    used in the field of bioinformatics without the users having to understand the 
    biological meaning of the data.
    \item \textbf{Unique Solution}: The property of convex optimization allows SVM
    to deliver unique solutions to one single dataset. For other algorithms, 
    such as Convolutional Neural Network, the result may vary based on the position of
    the datasets, since it alters the local minimum of the function.
    \item \textbf{Flexibility in Kernel Choice}: Even though we use the polynomial
    kernel and the radial basis kernel in most occasions, the choice of kernel is
    free to users, which means that users can use their own knowledge of the data
    to transform in the data in the way that it suits the real life situation. In this
    case, the SVM model will be fine-tuned in a way that it delivers an overall better
    result.
\end{itemize}

Like every other machine learning algorithm though,
SVM does have its disadvantages, even though it is widely used in many
fields. The main disadvantage is that the result of SVM is not 
interpretable and transparent enough. The relationship between the data
and the result is hard to explain, since the data has experienced 
complicated transformations through the kernel function.
Therefore, it would be hard to be used to explain
the relationship between the data and the result. For instance,
in the case of face recognition, SVM is able to distinguish
a person from another, but it cannot tell the user why it is
able to do so in such an accurate way. In this case, however, 
we are able to use data visualization to graphically interpret the relationship
between the input and the output. Bi-dimensional graphs with color coding are, 
for instance, used popularly in this case, since it shows the property of
the data distribution in a relational way. 
\cite{procon}
